% Copyright 2016 - 2017 Bas van Meerten and Wouter Franssen
%
%This file is part of ssNake.
%
%ssNake is free software: you can redistribute it and/or modify
%it under the terms of the GNU General Public License as published by
%the Free Software Foundation, either version 3 of the License, or
%(at your option) any later version.
%
%ssNake is distributed in the hope that it will be useful,
%but WITHOUT ANY WARRANTY; without even the implied warranty of
%MERCHANTABILITY or FITNESS FOR A PARTICULAR PURPOSE.  See the
%GNU General Public License for more details.
%
%You should have received a copy of the GNU General Public License
%along with ssNake. If not, see <http://www.gnu.org/licenses/>.

\documentclass[11pt,a4paper]{article}
\include{DeStijl}

\usepackage[bitstream-charter]{mathdesign}
\usepackage[T1]{fontenc}
\usepackage[protrusion=true,expansion,tracking=true]{microtype}
\pgfplotsset{compat=1.7,/pgf/number format/1000 sep={}, axis lines*=left,axis line style={gray},every outer x axis line/.append style={-stealth'},every outer y axis line/.append style={-stealth'},tick label style={font=\small},label style={font=\small},legend style={font=\footnotesize}}
\usepackage{colortbl}
\usetikzlibrary{calc}

%Set section font
\usepackage{sectsty}
\allsectionsfont{\color{black!70}\fontfamily{SourceSansPro-LF}\selectfont}
%--------------------


%Set toc fonts
\usepackage{tocloft}
%\renewcommand\cftchapfont{\fontfamily{SourceSansPro-LF}\bfseries}
\renewcommand\cfttoctitlefont{\color{black!70}\Huge\fontfamily{SourceSansPro-LF}\bfseries}
\renewcommand\cftsecfont{\fontfamily{SourceSansPro-LF}\selectfont}
%\renewcommand\cftchappagefont{\fontfamily{SourceSansPro-LF}\bfseries}
\renewcommand\cftsecpagefont{\fontfamily{SourceSansPro-LF}\selectfont}
\renewcommand\cftsubsecfont{\fontfamily{SourceSansPro-LF}\selectfont}
\renewcommand\cftsubsecpagefont{\fontfamily{SourceSansPro-LF}\selectfont}
%--------------------

%Define header/foot
%\usepackage{fancyhdr}
%\pagestyle{fancy}
%\fancyhead[LE,RO]{\fontfamily{SourceSansPro-LF}\selectfont \thepage}
%\fancyhead[LO,RE]{\fontfamily{SourceSansPro-LF}\selectfont \leftmark}
%\fancyfoot[C]{}
%--------------------

%remove page number from first chapter page
%\makeatletter
%\let\ps@plain\ps@empty
%\makeatother
%----------------------

\usepackage[hidelinks,colorlinks,allcolors=black, pdftitle={DQSQ},pdfauthor={Wouter M.J.\ Franssen}]{hyperref}

\interfootnotelinepenalty=10000 %prevents splitting of footnote over multiple pages
\linespread{1.2}

\title{\color{black}\fontfamily{SourceSansPro-LF}\bfseries 2D double-quantum/single-quantum (SQ/DQ)
correlation spectroscopy analysis in ssNake}
\author{}
\date{\color{black}\fontfamily{SourceSansPro-LF}\bfseries \today}


\begin{document}
%\newgeometry{left=72pt,right=72pt,top=95pt,bottom=95pt,footnotesep=0.5cm}
\microtypesetup{protrusion=true} % enables protrusion

\maketitle

\section{Introduction}
The following will explain how 2D double-quantum single-quantum NMR data can be processed in ssNake.
 The
tutorial delivered with the ssNake program is considered as prior knowledge. If you have not yet
studied this, please do so before continuing with this example.

In liquid state NMR, homonuclear correlations (e.g.\ $^{1}$H-$^1$H) are commonly recorded using a
COSY experiment. This experiment uses the scalar couplings (J-couplings) between the different nuclei
to transfer magnetization between them, and to gain information into the connectivity between the
spins. Another way to do this, is to use dipolar couplings, and go for a NOESY experiment. In solid
state systems, scaler couplings are often to weak for COSY experiments (as the relaxation times are
much shorter), and other experiments need to be considered. A common 2D homonuclear solid state
experiment is a double-quantum single-quantum experiment. In this experiment, the direct dimension
is the single-quantum dimension, and the indirect dimension the double quantum. 

If two spins A and B are connected (via a dipolar coupling), they will show crosspeaks in the DQSQ
experiment. In the DQ dimension, A and B can have a peak at their double frequency (because
of the double quantum character). If there is a sizable dipolar coupling between A and B, there will
also be crosspeaks at the sum shift $\delta_\text{A} + \delta_\text{B}$. The intensity of
these peaks compared to the main (diagonal) peaks gives information on the strength of the dipolar
coupling.

\section{Data}
The data we will use in this tutorial is from a $^1$H--$^1$H DQSQ experiment on methylammonium lead iodide
(CH$_3$NH$_3$PbI$_3$), a perovskite material used in a new generation of solar cells. The spectrum
was recorded on a Varian 400 MHz machine, using a 3.2 mm rotor and 15 kHz MAS.



\section{Processing}
Processing a DQSQ experiment is mostly just regular 2D data processing. Only at the stage of
referencing do things get a bit tricky.

\begin{itemize}
  \item Open the data delivered with this tutorial in ssNake via Open $\longrightarrow$ File
	\item Zero fill the data to 4096 points (Matrix $\longrightarrow$ Sizing, and fill in 4096 at the
	  size and leave the offset as is)
	\item Push the `Fourier' button to go to the spectrum
	\item Phase the spectrum using only zero order phasing
\end{itemize}
This should give something like this:
\begin{center}
\includegraphics[width=0.7\linewidth]{Figs/Fig1.png}
\end{center}
And zoomed on the main peaks:
\begin{center}
\includegraphics[width=0.7\linewidth]{Figs/Fig2.png}
\end{center}
This shows the first trace of our 2D data. We have now processed the direct dimension (D2), we now
continue with the indirect dimension (D1).

\begin{itemize}
  \item Get the position of the rightmost main peak (so that we can view it along D1). Use `Get
	 Position' from the bottom frame, and click on the peak. This give x-Position = 2030 (or
	 something close to that).
  \item Fill in this value at the D2 box in the side frame and press `Enter'
\end{itemize}
This should give:
\begin{center}
\includegraphics[width=0.7\linewidth]{Figs/Fig3.png}
\end{center}
This data was recorded with a hypercomplex scheme (states--TPPI), and requires a conversion:

\begin{itemize}
  \item Use Tools $\longrightarrow$ Hypercomplex $\longrightarrow$ States--TPPI to convert it
	\item Use Tools $\longrightarrow$ Complex conjugate to invert the sense of direction to the
	  ssNake definition \footnote{Most Varian sequences define the sense of the rotation in the
	  indirect dimension in the opposite way as we do it. This is, however, dependent on the way the
	used pulse-sequence is written, so we cannot correct this for you automatically. Without this
 step, the spectrum along D1 will appear flipped. }
\end{itemize}
This should give:
\begin{center}
\includegraphics[width=0.7\linewidth]{Figs/Fig4.png}
\end{center}

\begin{itemize}
  \item Zero fill to 512 points
  \item Fourier transform
  \item Phase (0 order only)
\end{itemize}
This gives:
\begin{center}
\includegraphics[width=0.7\linewidth]{Figs/Fig5.png}
\end{center}

We now need to set the correct chemical shift reference for both dimensions. The reference frequency
(0 ppm) is  399.9344480 MHz (based on an external reference). We can immediately apply this to the
direct dimension (D2):


\begin{itemize}
  \item Go to D2 (use the radio button in the side frame)
  \item Use Plot $\longrightarrow$ Reference $\longrightarrow$ Set Reference, and fill in
	 399.9344480 for the frequency (and leave the other values unchanged).
	\item Set the current view to `ppm' (in the bottom frame with `Axis')
\end{itemize}
This gives:

\begin{center}
\includegraphics[width=0.7\linewidth]{Figs/Fig6.png}
\end{center}

Now we continue with D1. The tricky part with referencing a double-quantum dimension is that our
external reference is not correct anymore: all frequencies are doubled in the indirect dimension.
This means that our zero ppm frequency (399.9344480 MHz) is different in D1: its difference with
respect to the carrier frequency should be doubled. In our case, the carrier frequency (centre of
the spectrum) is 399.936414 MHz (see the bottom frame). The difference with this and the reference
frequency is -0.0019659 MHz. This value should be added to the reference frequency (this makes the
new reference frequency to be twice as far away from the centre frequency as before). This makes to
reference frequency in D1 equal to  $399.9344480 -0.0019659 = 399.9324820$ MHz.

\begin{itemize}
  \item Go to D1 (use the radio button in the side frame)
  \item Use Plot $\longrightarrow$ Reference $\longrightarrow$ Set Reference, and fill in
399.9324820 MHz
	 	\item Set the current view to `ppm' (in the bottom frame with `Axis')
	\item Go back to D2 (radio button)
\end{itemize}
Now, we are finished with the processing, and it is time to view our 2D spectrum. Use Plot
$\longrightarrow$ Contour to make a contour plot. Zoomed in, this looks like (with the lower contour
level at 55 \%, see the side frame, contour levels can be zoomed with `shift + scroll') :

\begin{center}
\includegraphics[width=0.7\linewidth]{Figs/Fig7.png}
\end{center}
This shows that the two peaks are connected. Both have an average peak in D1 at $\approx 10$ ppm, as
well as self peaks at twice there original chemical shift. These self peaks are located on the
diagonal, which ssNake can also draw. In the side frame enable the `Diagonal' and set the
multiplier at 2 (we want the diagonal to go twice as fast in D1, due to the double-quantum
character). This gives:

\begin{center}
\includegraphics[width=0.7\linewidth]{Figs/Fig8.png}
\end{center}
Which is the final spectrum. The pattern that is shown here is common for a DQSQ experiment, if the
two chemical species are connected via a dipolar coupling. The final spectrum is also delivered with
this tutorial (as a .mat file). Note that in the D2 dimension the width has been reduced to the
relevant part via Matrix $\longrightarrow$ Extract Part. This was done to reduced the saved data
size.



\end{document}
